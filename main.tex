\documentclass[conference]{IEEEtran}
\IEEEoverridecommandlockouts
% The preceding line is only needed to identify funding in the first footnote. If that is unneeded, please comment it out.
\usepackage{cite}
\usepackage{amsmath,amssymb,amsfonts}
\usepackage{algorithmic}
\usepackage{graphicx}
\usepackage{textcomp}
\usepackage{xcolor}
\usepackage{csquotes}
\usepackage{float}
\def\BibTeX{{\rm B\kern-.05em{\sc i\kern-.025em b}\kern-.08em
		T\kern-.1667em\lower.7ex\hbox{E}\kern-.125emX}}
\begin{document}
	
	\title{\textit{Diabetes Mellitus} e Glicômetro}
	
	\author{\IEEEauthorblockN{1\textsuperscript{st} Leonardo Held}
		\IEEEauthorblockA{\textit{Departamento de Engenharia Elétrica e Eletrônica} \\
			\textit{Universidade Federal de Santa Catarina}\\
			Capinzal, Brasil \\
			leonardo.held@grad.ufsc.br}
		\and
		\IEEEauthorblockN{2\textsuperscript{nd} Gustavo Batistell}
		\IEEEauthorblockA{\textit{Departamento de Engenharia Elétrica e Eletrônica} \\
			\textit{Universidade Federal de Santa Catarina}\\
			Florianópolis, Brasil \\
			gustavo.batistell@grad.ufsc.br}
				\and

		\IEEEauthorblockN{3\textsuperscript{nd} Guilherme Augusto da Silva	}
		\IEEEauthorblockA{\textit{Departamento de Engenharia Elétrica e Eletrônica} \\
			\textit{Universidade Federal de Santa Catarina}\\
			Florianópolis, Brasil \\
			g.augusto.s@grad.ufsc.br}
		}
	
	\maketitle
	
	\begin{abstract}
		O presente seminário busca descrever a tecnologia e princípio de funcionamento do glicômetro, dispositivo utilizado para medir a concentração aproximada de glucose no sangue. O seminário também dá um enfoque nos processos biológicos que causam a \textit{diabetes mellitus} e a hipoglicemia associada.
	\end{abstract}
	
	\begin{IEEEkeywords}
		Glicosímetro, Glicômetro, Diabetes Mellitus, Metrologia, Calibração, Engenharia Biomédica, Revisão Bibliográfica
	\end{IEEEkeywords}
	
	\section{Introdução}
	Glicômetros fazem parte parte do dia-a-dia de uma parcela significativa da população e estão entre os equipamentos de medição de níveis biológicos mais utilizados no mundo. De acordo com a \textit{Internacional Diabetes Foundation}, a Diabetes Mellitus tem prevelência em pelo menos 9,3\% da população mundial sendo acometida pela doença, com mais da metade dos adultos não diagnosticados\cite{IDF}. 
	
	O presente seminário busca, de forma sucinta, relevar os principais pontos sobre os mecanimos de ação e a engenharia necessária por trás do aparelho Glicômetro, bem como fazer um estudo superficial da causa da Diabetes e o processo biológico associado.
	
	\section{Processo Biológico}
	
	\subsection{Descrição da doença}
	
	A \textit{Diabetes mellitus} é uma doença crônica do metabolismo de carboidratos que resulta na produção insuficiente do hormônio insulina, ou também na utilização incorreta deste nos processos biológicos por parte das células.
	
	A insulina é normalmente produzida em células $\beta$ em estruturas denominadas \textit{Ilhas de Langerhans} ou \textit{Ilhas Pancreáticas}. A insulina gerada por essas células é utilizada para possibilitar a sua absorção por células no corpo e consequentemente diminuir a concentração de glucose no sangue.\cite{beta}
	
	Quando as células $\beta$ possuem deficiência de produção de insulina, a concentração de glucose no sangue sobe para níveis elevados, gerando a condição de \textit{hiperglicemia}. Sem a glucose, as células do corpo começam a metabolizar as reservas de gordura e proteínas, que em torno geram altos níveis de \textit{ketones} e \textit{keto-ácidos} como biproduto, causando uma sobrecarga desses compostos e consequentemente uma condição denominada \textit{diabetes ketoacidose}.\cite{diseases}
	
	Juntas, a hiperglicemia e a ketoacidose são responsáveis pela doença chamada diabetes mellitus.
	
	\section{Curta introdução aos tipos de Diabetes Mellitus}
	\subsection{Diabetes Tipo I}
	Conhecida como \textit{diabetes dependente de insulina}, acredita-se que esse tipo de \textit{diabetes mellitus} é causada por uma condição auto-imune que em torno destrói a produção de células $\beta$ no corpo, decorrentemente deprivando o corpo da capacidade de regulação da concentração de glucose no sangue.
	
	O corpo destrói cerca de $75\%$ das células $\beta$. O paciente, com auxílio de administração regular de doses de insulina e controle alimentar rígido pode ter os sintomas controlados.\cite{types}
	\subsection{Diabetes Tipo II}
	Mais comum do que o Tipo I, a diabetes Tipo II tem forte correlação com a idade. Geralmente a partir dos 45 anos, apesar do pâncreas ainda ter capacidade de produção de secretar insulina via as células $\beta$, o corpo se torna naturalmente resistente contra a insulina.
	
	Como a sinalização de que o hormônio está sendo aproveitado de maneira satisfatória existe de maneira fraca pela pouca absorção e contínua concentração de glucose no sangue, a produção de insulina aumenta causando cansaço nas células que a produzem, e eventual falha na produção com o passar dos anos.\cite{types}
	
	\subsection{DMG - Diabetes Mellitus Gestacional\cite{gestacional}}
	Existe chance de uma pessoa biologicamente mulher desenvolver diabetes durante a gestação. O mecanismo de causa é pouco conhecido, e os estudos revisados no presente seminário são inconclusivos quanto à porcentagem da população estudada.
	
	O quadro clínico geralmente se resolve após o parto.
	
	\subsection{Diagnóstico}
	A \textit{diabetes mellitus} é diagnosticada via teste de níveis de concentração de glucose no sangue. A \textbf{OMS}, - Organização Mundial da Saúde, define guias que implicam na condição se qualquer uma das seguintes condições forem contempladas\cite{who}:
	
	\begin{itemize}
		\item Nível de glucose no plasma em jejum $\geq$ 7.0 $\frac{\text{mmol}}{\text{L}}$.
		\item Nível de glucose no plasma depois de duas horas de um teste de tolerância de glucose (TTG) $\geq$ 11.1 $\frac{\text{mmol}}{\text{L}}$.
		\item Quadro de sintoma de alta concentração de glucose no sangue e $\geq$ 11.1 $\frac{\text{mmol}}{\text{L}}$.
		\item Hemoglobina glicosilada $\geq$ 48 $\frac{\text{mmol}}{\text{mol}}$.
	\end{itemize}

\begin{table}[]
	\begin{tabular}{|l|l|l|l|}
		\hline
		Condição & Teste Glucose & Jejum & Hemoglobina glicosilada  \\ \hline
		Unidade &  {mmol}/{L} & {mmol}/{L}  & {mmol}/{mol} \\ \hline
		Normal & $<$ 7.8 & 6.1 & $<$ 42 \\ \hline
		Jejum & $<$ 7.8 & 6.1 - 7.0 & 42 - 46 \\ \hline
		TTG & $\geq$ 7.8 & $<$ 7.0 & 42 - 46 \\ \hline
		Diabetes mellitus & $\geq$ 11.1 & $\geq$ 7.0 & $\geq$ 6.5 \\ \hline
	\end{tabular}
\begin{center}
	\footnotesize{Tabela 1: Condições e níveis de glucose associados para diagnóstico da Diabetes mellitus.  Retirado de Definition and diagnosis of diabetes mellitus and intermediate hyperglycemia: Report of a WHO/IDF consultation.}
\end{center}
\end{table}

\section{Glicômetro e a detecção da concentração de glucose no sangue}
\subsection{Introdução da tecnologia e background histórico}
	Glicômetros medem a concentração aproximada de glucose no sangue por diferentes mecanismos explorados nas seções seguintes. A tecnologia começou a ser desenvolvida na década de 50, com o aparecimento de biossensores e a teoria de eletrodos de oxigênio. O paper inicial e mais influente da área pode ser tomado como \textit{ELECTRODE SYSTEMS FOR CONTINUOUS MONITORING IN CARDIOVASCULAR SURGERY}\cite{clark}, de 1962, por Leland Clark e Champ Lyons, e se fundamentava no princípio de uma camada de glicose oxidase sob um eletrodo de oxigênio. 
	
	Os primeiros modelos comerciais disponíveis para o público geral foram demonstrados em 1971 e vendidos pela Dextrostix, e se chamavam "Glucometer", nome de produto que se tornou homônimo com a tecnologia anos depois. Originalmente, os testes eram realizados apenas em consultórios médicos.\cite{history}
	
	Vale notar que toda a tecnologia inicial foi desenvolvida para detecção de diabetes tipo I, e apesar de existir glicômetros para diabetes tipo II, uma massiva maioria das pessoas que possuem a condição não são instruídas a utilizar o equipamento.
	
	Vários tipos de monitores foram desenvovidos ao longo dos anos. Os principais são os portáteis, utilizando fitas de teste descartáveis que contém a glicose oxidase necessária para o teste, laboratoriais, com esquemas de monitoramento e \textit{bookkeeping}, e os de monitoramento contínuo, que utilizam uma ponta de teste permanentemente atrelada ao paciente.
	
\section{Descrição da tecnologia}
\begin{figure}
	\centering
	\includegraphics[scale=0.2]{glocumeter.png}
	\caption{Gráfico demonstrando a utilização do glicômetro portátil com fita de teste.}
	\medskip
	\small
	
\end{figure}
\subsection{Princípio da Detecção da Concentração de Glucose}
A detecção da concentração ocorre por duas etapas: uma reação eletroenzimática com um catalisador e a detecção. 

A porção enzimática geralmente se dá pela presença de Glucose com algum pó enzimático. A parte enzimática pode ser integrada em pequenas faixas de teste (Fig. 1) descartáveis. O paciente realiza uma pequena perfuração no dedo, depositando uma pequena quantidade de sangue na faixa de teste. A Glucose reidrata o reagente enzimático, e dependendo de qual utilizado, mecanismos diferentes de detecção são utilizados.

Diversos compostos enzimáticos foram descobertos e são utilizados para a realização da detecção. O mais comum deles é a Glucose Oxidase (GOx). Na presença de Glucose (mais especificamente $\beta$-D-glucose), uma reação de oxiredução acontece, com a Glucose ($C_{6}H_{12}O_{6} + O_{2}$) reagindo com oxigênio formando Gluconato D-glucono-$\delta$-lactone e peróxido de hidrogênio. 

\[C_{6}H_{12}O_{6} + O_{2} \xrightarrow[]{\text{Glucose Oxidase}} \text{Gluconato} + H_{2}O_{2} \]

Nos primódios, a detecção era realizada via colorimetria, variante com a concentração de $H_{2}O_{2}$ acumulada durante o desbalanço da equação da esquerda para a direita e a redução em pH que acontece devido a quebra do D-glucono-$\delta$-lactone. Aliados com um reagente de cor, ambos métodos eram utilizados para detecção da concentração de uma forma manual, mas de fato mais subjetiva e menos precisa.

Como é uma reação de oxiredução, existe troca de elétrons, e determinado corretamente os potenciais de referência, uma correlação pode ser estabelecida entre a corrente determinada pela reação e a concentração de glucose no sangue.

Os materiais de referência\cite{microchip} \cite{nxp} concordam na utilização de um transdutor de corrente para tensão, filtragem do sinal e então um conversor analógico digital para a amostragem da tensão. Geralmente também existe um display LCD/LED que disponiliza a informação de forma visual para o usuário (Fig. 1).

A fita de testes deve conter, desta forma, três terminais básicos em forma de eletrodo, denominados Referencial, Contador ou Gatilho e eletrodo de Trabalho\cite{strips}: 

\begin{figure}
	\centering
	\includegraphics[scale=0.2]{strip.png}
	\caption{Uma fita de teste com seus terminais. Retirado de \cite{strips}}
	\medskip
	\small
	
\end{figure}

Um circuito básico de transimpedância é desenhado na Fig. 3, de acordo com a literatura supracitada, com uma tensão negativa sendo aplicada no terminal de referência da fita de teste e uma tensão precisa de referência aplicada no amplificador operacional. Um filtro básico de primeira ordem é aplicado na saída. A equação de saída pode ser dada por

\[V_{ADC} = -(\dfrac{{V_{Eletrodo Contador}} - {V_{ref}}}{R_{fita}}) \cdot R \]

com frequência de corte em

\[f = \dfrac{1}{2 \cdot \pi \cdot R_{2} \cdot C_{2}}\]


\begin{figure}[H]
	\centering
	\includegraphics[scale=0.2]{refdesign.png}
	\caption{Um design de referência de amplificador de transimpedância conectado à uma fita de teste.}
	\medskip
	\small
	
\end{figure}

A parte microcontrolada e minúcias de amostragem estão fora do escopo desse seminário, porém, vale notar que é um sistema simples, que não requer respostas em tempo real\footnote{O ADC também deve ser calibrado com uma série de referência para concentrações calculadas de Glucose presentes na fita. Isso deve ser realizado para cada equipamento individual.}

\subsection{Normas Técnicas Associadas}
Não foram encontradas normas técnicas associadas à medição de concentração de glicose no sangue no âmbito do presente seminário\cite{castro}. Internacionalmente, a calibração é regulamentada pela \textit{ISO 15197:2013 In vitro diagnostic test systems — Requirements for blood-glucose monitoring systems for self-testing in managing diabetes mellitus}\cite{iso}. No Brasil não existe exigência de aplicação dessa ou norma de natureza semelhante\cite{infoglico}. 

De qualquer forma, como equipamentos médicos e de usuário, esses equipamentos são regulados pelo Inmetro via normas com menor especificidade, se baseando na NBR IEC 60601-1 (1994), dado que é um equipamento médico, a ISO/IEC GUIDE37:2012 \textit{Instructions for use of products by consumers} e a ISO 9355:2009 \textit{Ergonomic requirements for the design of display s and control actuators}, dado que a maioria dos dispositivos no mercado hoje possuem telas.\cite{inmetro}

\subsection{Características de Desempenho e Métricas de Avaliação}
Os fabricantes dos glicosímetros disponíveis no mercado oferecem fitas/tiras padrão e soluções de controle com concentrações conhecidas para que os usuários te	nham a possibilidade de conferir o bom funcionamento do seu equipamento de monitoramento do índice glicêmico, mas isso não impede a venda de produtos com desempenho inferior ao especificado mesmo de marcas bastante difundidas. Esse fato sobre o desempenho desses equipamentos se confirma, por exemplo, em publicação da Anvisa\cite{anvisa}, onde são listados 16 modelos de glicosímetros de várias marcas que tiveram seus registros cancelados por apresentarem resultados em desacordo com a ISO 15197:2013, que estabelece uma variação máxima de 15\% dos valores medidos em relação aos obtidos em laboratório. \\
Outra dificuldade é a padronização e fiscalização do desempenho desses equipamentos, pois apesar da norma ISO 15197:2013 estabelecer os requisitos mínimos, não é possível encontrar na página da Rede Brasileira de Calibração (RBC-INMETRO), laboratórios certificados para realização de calibrações ou ensaios em glicosímetros. De fato, na página do INMETRO, é possível encontrar uma publicação curta de 01/12/2017, atualizada em 29/07/2021, que diz:

\blockquote{“Informamos que o produto/instrumento denominado Glicosímetro não está contemplado por Regulamentação Técnica Metrológica. Sendo assim, tal produto não é objeto de aplicação do Controle Metrológico Legal, não havendo obrigatoriedade de submissão à Avaliação de Modelo e/ou às verificações no escopo da Metrologia Legal.”\cite{glicoinfo}}

\subsection{Comparação de dados entre modelos de fabricantes}
A grande gama de modelos e fabricantes de glicosímetros disponíveis no mercado dificultam a comparação. Por exemplo na publicação\cite{minas} da Revista Médica de Minas gerais, até mesmo modelos de linhas similares do mesmo fabricante apresentam resultados discrepantes comparando dados com amostras analisadas em laboratório, onde, o Accu-Chek Performa, por exemplo, apresentou resultados médios significantemente diferentes em relação a análise laboratorial enquanto Accu-Chek Advantage apresentou resultados muito próximos com alta correlação.
Em geral, outros tantos critérios comparativos que não o desempenho das medições podem ser levantados, desde o volume da amostra de sangue necessário na tira, a capacidade de memória, o tamanho e iluminação do display, o custo das tiras, a portabilidade e até mesmo modelos com sensores subcutâneos que minimizam consideravelmente os desconfortáveis furos com lanceta (punções).
\subsection{SMCGL - Automonitoramento de Glucose contínua e a comunidade hacker}
Atualmente, glicômetros de monitoramento contínuo comerciais - \textit{Self Monitoring Continous Glucose Levels} são equipamentos caros, com firmware propositalmente fechado e de difícil reparo. Relevando o custo, o uso de uma estratégia de dosa de insulina automática via um equipamento de controle de malha fechada dos níveis de glucose pode representar uma melhora significativa na qualidade de vida do paciente, além de ainda não serem totalmente aprovados pelos órgãos responsáveis.

Com isso, projetos como o OpenAPS (\textit{Open Artificial Pancreas System project}) tentam acelerar o desenvolvimento de aplicações de controle de malha fechada dos níveis de glucose via \textit{hacking} de bombas de insulina disponíveis no mercado, que normalmente precisam de feedback de usuário para o funcionamento, e normalmente apenas administram insulina em determinados horários, sem consideração para o estado atual do sistema, como nos equipamentos comerciais mais caros. Mais de $2,344$ indivíduos já fazem uso de sistemas de controle de insulina por malha fechada desenvolvidos sem intervenção de empresas (DIY), de acordo com a OpenAPS.

Há, claramente, uma necessidade de mercado que não foi corretamente atendida por completo, seja por intervenções governamentais demoradas ou pela falta de ação de companhias privadas, que gerou uma reação do crescente levante de pessoas conscientes do uso de tecnologia fechada nas suas vidas.

Por outro lado, a falta de regulamentação e a experimentação muitas vezes levada pelo desespero pode acarretar em danos irreparáveis à saúde do indivíduo.
\section{Breve Conclusão}
A capacidade de compreender um fenômeno e transladá-lo para um domínio de conhecimento próximo (por exemplo, passar da reação química até uma tela que mostra a concentração de glucose no sangue) é uma habilidade essencial para qualquer engenheiro que queira trabalhar numa área tão multidisciplinar e com tanta necessidade de integração.

Nesse âmbito, pode-se concluir que o presente seminário teve sucesso no desenvolvimento dessa capacidade e no estudo de como obter material relevante em Engenharia Biomédica, além de por em exercício mental e argumentativo das várias das habilidades desenvolvidas durante o curso.


	\begin{thebibliography}{00}
		\bibitem{diseases} Tamparo, Carol D., and Marcia A Lewis. Diseases of the Human Body. 5th ed. Philadelphia, PA: F.A. Davis, 2011.
		\bibitem{beta} Chen C, Cohrs CM, Stertmann J, Bozsak R, Speier S. Human beta cell mass and function in diabetes: Recent advances in knowledge and technologies to understand disease pathogenesis. Mol Metab. 2017;6(9):943-957. Published 2017 Jul 8. doi:10.1016/j.molmet.2017.06.019
		\bibitem{types} Butler A E, Misselbrook D. Distinguishing between type 1 and type 2 diabetes BMJ 2020; 370 :m2998 doi:10.1136/bmj.m2998
	    \bibitem{gestacional} https://www.nhs.uk/conditions/gestational-diabetes/. Acessado em 01 de Agosto de 2021.
	    \bibitem{who}Alberti, K G, and P Z Zimmet. “Definition, diagnosis and classification of diabetes mellitus and its complications. Part 1: diagnosis and classification of diabetes mellitus provisional report of a WHO consultation.” Diabetic medicine : a journal of the British Diabetic Association vol. 15,7 (1998): 539-53. doi:10.1002/(SICI)1096-9136(199807)15:7<539::AID-DIA668>3.0.CO;2-S
	    \bibitem{clark} Clark, L.C., Jr. and Lyons, C. (1962), ELECTRODE SYSTEMS FOR CONTINUOUS MONITORING IN CARDIOVASCULAR SURGERY. Annals of the New York Academy of Sciences, 102: 29-45. https://doi.org/10.1111/j.1749-6632.1962.tb13623.x
	    \bibitem{history} History of Glucose Monitoring. Irl B. Hirsch, MD, University of Washington, Seattle, WA 2018
	    \bibitem{openaps} https://openaps.org/. Acessado em 01 de Agosto de 2021.
	    \bibitem{castro} CASTRO JúNIOR, Roberto. Glicosímetro de pulso. 2010. Tese (Doutorado em Sistemas Eletrônicos) - Escola Politécnica, Universidade de São Paulo, São Paulo, 2010. doi:10.11606/T.3.2010.tde-16082010-161914. Acesso em: 2021-08-01.
	    \bibitem{inmetro} http://www.inmetro.gov.br/consumidor/produtos/glicosimetro.pdf. Acessado em 01 de Agosto de 2021.
	    \bibitem{strips}Hsin-Yi Kuo, Yuan-Hwei Cheng, Ho Chang, Jin-Siang Shaw, Rahnfong Lee, "Design of Electrodes on Gold Test Strips for Enhanced Accuracy in Glucose Measurement", Journal of Sensors, vol. 2019, Article ID 8627198, 10 pages, 2019. https://doi.org/10.1155/2019/8627198
	    \bibitem{microchip} Glucose Meter Fundamentals and Design Application Note Rev. 1, 01/2013 NXP
	    \bibitem{nxp} Glucose Meter Reference Design AN1560 Microchip
	    \bibitem{iso} ISO 15197:2013, In vitro diagnostic test systems — Requirements for blood-glucose monitoring systems for self-testing in managing diabetes mellitus
	    \bibitem{infoglico} https://www.gov.br/inmetro/pt-br/acesso-a-informacao/perguntas-frequentes/metrologia-legal/como-obter-informacoes-sobre-glicosimetro. Acessado em 02 de Agosto de 2021.
	    \bibitem{IDF} https://www.diabetesatlas.org/en/. Acessado em 02 de Agosto de 2021.
	    \bibitem{anvisa} Cancelado registro de 16 modelos de glicosímetros. https://www.gov.br/anvisa/pt-br/assuntos/noticias-anvisa/2018/cancelado-registro-de-16-modelos-de-glicosimetros. Acessado em 02 de agosto de 2021.
	    \bibitem{glicoinfo} Como obter informações sobre glicosímetro? https://www.gov.br/inmetro/pt-br/acesso-a-informacao/perguntas-frequentes/metrologia-legal/como-obter-informacoes-sobre-glicosimetro. Acessado em 02 de agosto de 2021.
	    \bibitem{minas} Comparação entre determinações de glicemia capilar e venosa com glicosímetros e dosagem laboratorial da glicose plasmática venosa. http://www.rmmg.org/artigo/detalhes/2054. Acessado em 02 de agosto de 2021.
 	\end{thebibliography}

	
\end{document}